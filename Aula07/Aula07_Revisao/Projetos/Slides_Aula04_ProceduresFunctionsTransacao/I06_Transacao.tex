	\subsection{Transações}

	\begin{frame}
	
	
	
	
	\begin{CaixaModelo01}{Transação}
	
	\begin{itemize}
		\item 		Uma transação no banco de dados consiste em um conjunto de querys
		que alteram as informações de uma ou varias tabelas em mais de um registro,
		de tal forma que todas as alterações sejam bem sucedidas para a operação seja
		realizada com sucesso.
		\item Uma transação é dita ser uma operação atômica, ou seja,  todas as operações devem ser bem sucedidas, 
		se não todas as operações devem ser desfeita.
		
		\item \textbf{COMMIT} -> bem sucedidas
		
		\item \textbf{ROOLBACK} -> mal sucedida
		
	\end{itemize}
		
		
%		
%		Uma transação é dita ser uma operação atômica, ou seja,  todas as operações devem ser bem sucedidas, 
%		se não todas as operações devem ser desfeita.
%		
%		
%		Se as operações forem bem sucedidas  realiza-se um \textbf{COMMIT}, caso contrário
%		deve-se realizar um \textbf{ROOLBACK} para desfazer tudo o que já foi realizado.
		
		
	\end{CaixaModelo01}

	\end{frame}