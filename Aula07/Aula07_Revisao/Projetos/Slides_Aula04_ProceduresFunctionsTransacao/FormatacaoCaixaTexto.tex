%% \usepackage{varwidth}
%% Definición del entorno "miEjemplo"---------------------------------------
\newtcolorbox{CaixaModelo01}[2][]
{%
enhanced, 
skin=enhancedlast jigsaw,
attach boxed title to top left={xshift=-4mm,yshift=-0.5mm},
fonttitle=\bfseries\sffamily,
varwidth boxed title=0.7\linewidth,
colbacktitle=green!20!black,
colframe=green!50!black,
interior style={top color=green!10!white,bottom color=green!10!white},
boxed title style={empty,arc=0pt,outer arc=0pt,boxrule=0pt},
underlay boxed title={
\fill[green!70!black] (title.north west) -- (title.north east)
-- +(\tcboxedtitleheight-1mm,-\tcboxedtitleheight+1mm)
-- ([xshift=4mm,yshift=0.5mm]frame.north east) -- +(0mm,-1mm)
-- (title.south west) -- cycle;
\fill[green!45!white!50!black] ([yshift=-0.5mm]frame.north west)
-- +(-0.4,0) -- +(0,-0.3) -- cycle;
\fill[green!45!white!50!black] ([yshift=-0.5mm]frame.north east)
-- +(0,-0.3) -- +(0.4,0) -- cycle; },
title={#2}, 
#1}
%%-Usando el entorno "cajamiEjemplo"-----------------------------------------
% \newcommand{\sen}{{\rm sen}}
% \newcommand{\Co}{\mathbb{C}}